\section{Выводы}

В ходе выполнения лабораторных работ я собрал корпус на морскую тематику из двух источников (Wikipedia EN и MarineLink) и сделал простой поисковик по этому корпусу. 
Система умеет загружать документы, сохранять их, выделять из HTML чистый текст, делать токенизацию и стемминг, строить булев инвертированный индекс и выполнять булев поиск с операторами AND/OR/NOT и со скобками.
Также я сделал веб-страницу для поиска, чтобы было удобнее демонстрировать работу не только через консоль.

Главный плюс того, что получилось --- проект можно запускать по шагам: сначала собрать корпус, потом выделить текст, потом построить индекс и уже после этого выполнять запросы. Это удобно для тестирования и для того, чтобы не переделывать всё сразу при мелких изменениях.

Минусы текущей версии:
\begin{itemize}
    \item \textbf{Нет ранжирования.} Результаты поиска не сортируются по релевантности, поэтому в выдаче могут быть документы, которые формально подходят под запрос, но не являются лучшими ответами.
    \item \textbf{Дисбаланс источников.} Wikipedia значительно больше по количеству документов, из-за этого в выдаче она часто доминирует, и результаты из MarineLink могут появляться ниже (особенно если брать небольшой топ выдачи).
    \item \textbf{Простая токенизация.} Простые правила токенизации иногда дают неудачные токены (дефисы, апострофы, аббревиатуры), это может немного ухудшать поиск по некоторым запросам.
    \item \textbf{Стемминг не идеален.} Иногда разные слова могут сводиться к одному стему или наоборот, что может давать лишние совпадения.
\end{itemize}

Если улучшать систему дальше, то логичнее всего добавить ранжирование (например, TF-IDF или BM25), чтобы результаты были более полезными по порядку. Также можно улучшить обработку запросов: поддержать фразы, сделать более аккуратную токенизацию для дефисов и апострофов, а ещё добавить небольшие бонусы за точное совпадение слова без стемминга. 
В целом текущая версия закрывает требования для уровня ``удовлетворительно'' и показывает рабочий конвейер от скачивания документов до поиска по индексу.
